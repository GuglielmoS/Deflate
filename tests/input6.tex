\documentclass[11pt]{article}

% per utilizzare gli accenti
\usepackage[applemac]{inputenc}

% per utilizzare le immagini
\usepackage{graphicx}

% senza questo da alcuni problema sulla formattazione delle formule
\usepackage{amsmath}

% simboli insieme numerici
\usepackage{amssymb}

% impostazioni foglio
\usepackage[a4paper,top=1.5cm,bottom=1.5cm,left=2cm,right=2cm]{geometry}

% percorso cartella immagini
\graphicspath{{images/}}

% prima pagina

\title{\textbf{Formulario - Algebra Lineare II}}
\author{\textit{Guglielmo Fachini}}
\date{\textit{\today{}}}

\begin{document}

\maketitle

\section{Autovalori e autovettori}
Data $f: \mathbb{V}^n \rightarrow \mathbb{V}^n$, se $\vec x \in \mathbb{V}^n, \vec x \ne \vec 0$ allora:
\newline
$$ \vec y = F \vec x = \lambda \vec x, \lambda \in \mathbb{R} $$
Si dice quindi che $\vec x$ � un \textit{autovettore} per $F$ relativo all'\textit{autovalore} $\lambda$.  

\subsection{Calcolo di autovalori}
Gli autovalori vengono trovati ponendo uguale a zero il polinomio caratteristico $P_A(\lambda$)
\begin{equation}
P_A(\lambda) = det(A - \lambda I)
\end{equation}
\subsubsection{Propriet�}
\begin{itemize}
\item{gli autovalori di una matrice triangolare o diagonale coincidono con gli elementi sulla diagonale della matrice stessa}
\item{una matrice di dimensione $n$ x $n$, possiede sempre $n$ autovalori (reali o complessi)}
\item{l'insieme di tutti gli autovalori viene detto spettro e si indica con $S_A$}
\item{la molteplicit� algebrica di un autovalore indica quante volte questo si ripete nello spettro}
\item{la somma delle molteplicit� algebriche degli autovalori di una matrice $n$ x $n$ � uguale ad $n$}
\item{le matrici simmetriche possiedono tutti autovalori reali}
\end{itemize}

\section{Autospazi}
Data una matrice quadrata $A$ e un autovalore $\lambda$ di A, l'autospazio relativo a $\lambda$, indicato con
 $E_{\lambda}$, � formato da tutti gli autovettori relativi a $\lambda$ e dal vettore nullo. Si pu� verificare che gli
 autospazi sono spazi vettoriali.
 
 \subsection{Calcolo di autospazi}
Il calcolo dell'autospazio $E_\lambda$ relativo all'autovalore
 $\lambda$ si effettua risolvendo il sistema omogeneo:
 \begin{equation}
 (A - \lambda I)\vec x = \vec 0 
 \end{equation}

\subsubsection{Propriet�}
\begin{itemize}
\item{la molteplicit� geometrica di un autovalore $\lambda$, � pari al numero di vettori linearmente indipendenti
	che generano l'autospazio $E_\lambda$, ed � sempre minore o uguale alla molteplicit� algebrica di $\lambda$}
\item{autovettori relativi ad autovalori diversi di una stessa matrice, sono tra loro linearmente indipendenti}
\item{l'autospazio $E_0$ corrisponde a $ker(A)$, perci� se $m.a.(0) = 1$ ho che $rang(A) = n - 1$}
\end{itemize}

\newpage

\section{Autovalori ed autospazi di $A^{-1}$ e $A^n \; (n \in \mathbb{N)}$}
Sia $A_n$ con $S_A = \left\{ \lambda_1, \dots, \lambda_n \right\}$
\begin{enumerate}
\item{Se $\lambda_i \ne 0 ,\; \forall i \in \left\{1, \dots, n\right\}$ allora $\exists A^{-1} $ con 
	$S_A = \left\{ \frac{1}{\lambda_1}, \frac{1}{\lambda_2}, \dots, \frac{1}{\lambda_n} \right\} $}
\item{Per $A^m \; (m \in \mathbb{N})$ si ha $S_A = \left\{ \lambda_1^m, \lambda_2^m, \dots, \lambda_n^m \right\} $}
\end{enumerate}
Gli autospazi $E_{\lambda_i}$ rimangono invariati in tutti e due i casi.

\section{Matrici simili}
Due matrici $A_n$ e $B_n$ si dicono simili se esiste una matrice di passaggio $C_n$ non singolare ($\exists C_n^{-1}$) tale che:
$$ A = C B C^{-1} \Rightarrow B = C^{-1} A C $$
\subsection{Propriet�}
\begin{enumerate}
\item{$det(A) = det(B)$ \hfill (stesso determinante)}
\item{$tr(A) = tr(B)$ \hfill (stessa traccia)}
\item{$S_A = S_B$ \hfill (stessi autovalori)}
\item{Se $\vec x_i$ � un autovettore per $A$ relativo all'autovalore $\lambda_i$, allora $\vec y_i = C^{-1} \vec x_i$ �
	un autovettore per $B$ relativo all'autovalore $\lambda_i$}
\end{enumerate}

\section{Diagonalizzazione}
\`E un processo matematico attraverso il quale a partire da una matrice quadrata si ricava un matrice simile diagonale,
che ha gli autovalori della matrice di partenza sulla diagonale.
\newline
La condizione per cui una matrice pu� essere diagonalizzata � che la molteplicit� algebrica sia uguale alla molteplicit� geometrica per ogni autovalore:
$$ m.a.(\lambda_i) = m.g.(\lambda_i) $$
In questo modo, data una matrice quadrata $A$ che soddisfa questa condizione, possiamo dire che:
\begin{equation}
A = S \Lambda S^{-1}
\end{equation}
Dove $S$ � composta dagli autovettori di $A$, mentre $\Lambda$ � una matrice diagonale che ha per valori sulla diagonale gli autovalori di $A$.
\newline
Ad esempio:
$$ S_A = \left\{1,2\right\} , \; E_1 = \, <\left(\begin{array}{c}1 \\1\end{array}\right)> , \; E_2 = \, <\left(\begin{array}{c}0 \\1\end{array}\right)> $$
$$ \Rightarrow A = \left(\begin{array}{cc}1 & 0 \\1 & 1\end{array}\right) \left(\begin{array}{cc}1 & 0\\0 & 2\end{array}\right)
\left(\begin{array}{cc}1 & 0 \\1 & 1\end{array}\right)^{-1}  $$

\newpage

\section{Matrici simmetriche}
Sia $A_n$ una matrice simmetrica reale $(a_{ij} = a_{ji} , \; A^T = A, \; a_{ij} \in \mathbb{R})$ :
\begin{enumerate}
\item{tutti gli autovalori di $A$ sono reali \hfill $(\lambda_i \in \mathbb{R})$}
\item{$A$ � sempre diagonalizzabile \hfill $(m.a.(\lambda_i) = m.g.(\lambda_i), \: \forall \lambda_i)$}
\item{autovalori relativi ad autovettori distinti sono ortogonali \hfill (\textit{dot product} = 0)}
\item{a partire da $S$ si pu� costruire una matrice ortogonale $O$ che ha le colonne che sono ortogonali a due a due
	e di norma unitaria; inoltre $O^{-1} = O^T$. Per ottenere $O$ si costruiscono i versori degli autovettori.}
\end{enumerate}

\section{Teorema di Cayley-Hamilton}
Sia $A_n$ una matrice quadrata, allora $P_A(A)$ � un polinomio matriciale in $A$ di grado $n$.
\newline
Infatti conoscendo il polinomio caratteristico di $A$ ($P_A(\lambda)$), possiamo sostiuire $A$ al posto di $\lambda$,
andando cos� a creare questa condizione: 
\begin{equation}
P_A(A) = det(A - A I) = 0
\end{equation}
Da qui siamo quindi in grado di esprimere $A^n$, $A^{n+1}$, $A^{n+2}$ come combinazione lineare di $A^{n-1}$, $A^{n-2}$,
$\dots$, $A^2$, $A$, $I$.
\newline
Ad esempio:
$$ A = \left(\begin{array}{cc} -1 & 2 \\1 & -1\end{array}\right) $$
$$ P_A(\lambda) = det(A - \lambda I ) = \lambda^2 + 2 \lambda - 1$$
$$ P_A(A) = 0 \Rightarrow A^2 = -2 A + I $$
Ora posso scrivere $A^k$ come:
$$ A^k = A A^{k-1} $$

\section{Regola del determinante e della traccia}
Data una matrice $A$ di dimensione $n$ x $n$ e i suoi autovalori $\lambda_1,...,\lambda_n$, si ha che:
\begin{equation}
det(A) = \lambda_1 \cdot ... \cdot \lambda_n
\end{equation}
\begin{equation}
tr(A) = \lambda_1 + \dots + \lambda_n
\end{equation}
\subsubsection{Propriet�}
\begin{itemize}
\item{una matrice che possiede almeno un autovalore zero ha sicuramente determinante zero, perci� non
	� invertibile. Viceversa una matrice invertibile ha sicuramente gli autovalori diversi da 0}
\item{la traccia di una matrice reale � un numero reale, quindi la somma delle parti immaginarie di eventuali 
	autovalori complessi deve annullarsi}
\end{itemize}

\section{Calcolo di $e^A$ e $A^k$}
Data una matrice quadrata $A$ diagonalizzabile, posso calcolare $e^A$ ed $A^k$ come:
\begin{equation}
A^k = S \Lambda^k S^{-1}
\end{equation}
\begin{equation}
e^A = S e^\Lambda S^{-1}
\end{equation}

\newpage

\section{Autovalori e autovettori relativi a trasformazioni geometriche}
Quando si sta lavorando con delle trasformazioni geometriche � utile vedere come si comportano gli autovalori e
gli autospazi della rispettiva applicazione lineare.
\newline
\subsection{Proiezione ortogonale su un piano $\alpha$ passante per l'origine}
Per ogni vettore parallelo al piano $\alpha$, ho che dopo la propriezione rimarr� invariato. Perci� questo
mi indica che avr� un autospazio $E_1$ di due dimensioni che equivarr� ad $\alpha$, e il cui relativo autovalore
sar� 1. Dopo di che, sapendo che i vettori perpendicolari al piano finiranno nell'origine, posso anche stabilire che
esiste l'autospazio $E_0 = ker(P) = \left<\vec n_\alpha\right> $ il cui autovalore � 0.
\newline
Si avr� quindi: $S_P = \left\{0, 1, 1\right\} , \; det(P) = 0 , \; tr(P) = 2 $

\subsection{Omotetia in $\mathbb{R}^3$}
$$S_\Omega = \left\{k, k, k\right\} , \; E_k  = \mathbb{R}^3 $$

\subsection{Simmetria in $\mathbb{R}^3$, rispetto una retta passante per l'origine, con una rotazione di $\pi$}
$ S_S = \left\{-1, -1, 1\right\} , \; E_1 = $asse$ , \; E_{-1} = $ piano passante per 0 e $ \bot $ asse

\end{document}
